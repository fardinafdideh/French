\protect\hypertarget{2570}{}{}

{}

\begin{itemize}
\item
  \url{https://www.thoughtco.com/une-boite-vocabulary-1371651}
\item
  \url{https://vitrinelinguistique.oqlf.gouv.qc.ca/fiche-gdt/fiche/18047414/balade-a-pied}
\end{itemize}

\hfill\break

\hfill\break

Être sûr,~ avoir la certitude = to be sure~

Où (est)(se situe) = where is

En présentiel~

À distance~

Optionnel, facultatif = optional~

Un tâche (à effectuer) = task

Un défaut de connexion = connection loss, loss of connection~

ravitaillement = refueling

à la rescousse = to the rescue

everyday (adj.) = courante:

\begin{itemize}
\item
  everyday life = la vie courante
\end{itemize}

every day (adv.) = tous les jours

Every single day = chaque jour~

faire allusion = to allude, suggest or call attention to indirectly;
hint at.

\begin{itemize}
\item
  faire allusion au fait = to allude to the fact
\end{itemize}

poids = weight

les parties du corps = body parts:

\begin{itemize}
\item
  l'œil, les yeux
\item
  narines (f.) = nostrils
\item
  bouche (f.) = mouth
\item
  joue (f.) = cheek
\item
  {Sourcil = eyebrow~}
\item
  {cil (m.) = eyelash}
\item
  {front = forehead}
\item
  {menton = chin}
\item
  {les taches de rousseur = freckles}
\item
  {oreille = ear}
\item
  {orteil (m) = toe}
\item
  {carré, rond (}{visage)}
\item
  {cou = neck}
\item
  {épaule = shoulder}
\item
  {torse (m.),~}{depuis le cou jusqu'à la base du ventre} {= torso}
\item
  {Poitrine = chest}
\item
  {Poignet (m.) = wrist}
\item
  {ongle = nail}
\item
  {dos = back}
\item
  {ventre (m.) = belly, stomach, tommy}
\item
  {nombril = navel}
\item
  {pouce (m.) = thumb, inch}
\item
  {dessous de bras = armpit}
\item
  {avant bras = forearm}
\item
  {coude (m.) = elbow}
\item
  {*Hanche = hip}
\item
  {pelvis}
\item
  {cuisse = thigh}
\item
  {muscle du mollet = calf muscle}
\item
  {genou = knee}
\item
  {Cheville (f.) = ankle}
\item
  {plante du pied = sole (of the foot)}
\item
  {les poils = hairs}
\item
  {fémur,} {l'os le plus long, le plus lourd et le plus solide du corps
  humain.}{= femur}
\item
  {ligaments croisés = crossed/cruciate ligaments}
\end{itemize}

aussi longtemps que = as long as

COVID:

\begin{itemize}
\item
  quarantaine = quarantine
\item
  confinement = lockdown
\item
  pandémie = pandemie, pandemic
\item
  décret = decree
\item
  distanciation sociale = social distancing
\item
  écouvillon (Longue tige en métal ou en bois munie à son extrémité d'un
  morceau de coton ou de gaze, utilisée pour réaliser des prélèvements
  diagnostiques ou bien pour appliquer des produits antiseptiques ou
  analgésiques dans les cavités naturelles.) = swab
\item
  couvre-feu = curfew
\item
  être vacciné = to be/get vaccinated
\end{itemize}

An~hour ago:

\begin{itemize}
\item
  Colloquial: ça fait une heure~
\item
  Standard: il y a une heure~
\end{itemize}

"Ça fait" vs "il y a

Quai :

\begin{itemize}
\item
  Quai de la gare, quai de train = train platform~
\item
  Quai de gare (dans les gares et métros)
\item
  Dock
\item
  Wharf
\item
  The part of a city that runs along a river; riverbank~
\end{itemize}

Voie = track

dans le futur = in the future

prendre quelque jours de congés = to take a few days off

je ne suis pas Là = I am (not here)(away)

Ces jours-ci = these days

Quelque part = somewhere~

En vaut la peine = is worth it

Se reposer = to get (some) rest

Il le faut = it's necessary~

Course = race

Animal de compagnie = pet:

Chiot =~puppy~

Poisson rouge = goldfish

Perroquet = parrot~

tortue = tortoise, turtle

\begin{itemize}
\item
  tortue de mer = turtle
\end{itemize}

Niche = doghouse~

Laisse (f.) = leash

Collier = collar

Vétérinaire (m.) = vet

Nourrir = feed

Sortir = get out

Promener = to walk (pets)

Croquettes = kibble

Éduquer = to train

Chasser = to chase,~run after

Aboyer = to bark

Miauler = to meow~

Odieux(se), qui excite la haine, le dégoût = hateful

Saisissant (film/livre), qui surprend = striking, seizing, arresting,
gripping (movie/book)

\begin{itemize}
\item
  A gripping tv thriller
\end{itemize}

Loisirs = hobbies:

\begin{itemize}
\item
  Faire de la randonnée = hiking
\item
  (Durant)~Temps libre = (during)~free time
\item
  touT ton temps = all your time
\item
  Le wakeboard = wakeboarding~
\item
  {la planche à roulettes, le skateboard = skateboard}
\item
  La planche de surf = the surfboard~
\item
  La planche à voile = windsurfing board~
\item
  {voilier = yacht, sailboat, sailing ship}
\item
  {voile, Ce qui cache qqch. = veil}
\item
  {promenade en bateau = boat ride}
\item
  {Le masque de plongée = diving mask}
\item
  {au bord de la mer, en bord de mer = to the seaside, by the sea}
\item
  {Pinceau = brush~}
\item
  Toile = canvas~
\end{itemize}

grève = strike

Faire grève = to go on strike~

manifester = to protest (march on the streets with a banner), to join a
demonstration/protest on the streets

protester = protest (more broadly than manifester: to object to sth)

annulation = cancelation

Workers (travailleurs LESS common):

\begin{itemize}
\item
  Les~Ouvriers d'usine = factory workers~
\item
  Les employés de bureau = office workers~
\item
  Les ouvriers agricoles = farm workers~
\item
  Les~Chercheurs = research workers~
\end{itemize}

mieux:

mieux vaut-il~~= is it better

\begin{itemize}
\item
  il vaut mieux
\end{itemize}

faire au mieux = to do the best

au mieux = at best

en face à face = face to face

Pants:~

US:

\begin{itemize}
\item
  Pantalon~= pantaloons, pants
\item
  Knickerbocker, knicker = pantaloons that go down to slightly below the
  knees~
\item
  Socks = stockings for men
\end{itemize}

Pants (US) = trousers (BR)

Pants (BR) = underwear (underpants or panties particularly for women)
(US)

\hfill\break

Sont/aller/devenir de pire en pire = to get worse and worse~

Hotel:

\begin{itemize}
\item
  À l'hôtel : insists on location
\item
  Dans l'hôtel : insists on the space inside the hotel building
\end{itemize}

Cette robe ne me va/convient~pas bien = ... doesn't suit me well.

Baigner:

\begin{itemize}
\item
  Se baigner = (baigner comes from bain, but it doesn't mean to take a
  bath, nor related to the notion of cleaning/washing) relaxing or fun
  activity in a large water area (pool, lake, ocean, ...)
\item
  Baignoire = bathtub~
\end{itemize}

des médias sociaux = social media

\begin{itemize}
\item
  compte = account
\item
  {Publier, poster = to post (comment on a blog or social network, or
  sending a letter through the postal service)}
\item
  {légende = caption, legend}
\item
  {influenceur(euse) = influencer}
\item
  {s'abonner = to subscribe}
\item
  {aimer = to like}
\item
  {commenter = to comment}
\item
  {suivre = to follow}
\item
  {partager = to share}
\item
  {chercher = to search}
\end{itemize}

Anniversaire = anniversary, birthday

{il s'avère = it proves}

{ingérer = ingest}

{réguler = regulate}

{détenu = detained, imprisoned}

{pénitentiaire, Qui concerne les prisons, les détenus. = penitentiary,
prison}

{sifflement = whistling, hissing}

{siffler = to whistle}

{piètre, Très médiocre = poor}

{s'incorporer = incorporate}

{carence = deficiency, deprivation}

{excès = excess}

{arborisation =}

{Dessin naturel ressemblant à des végétations, à des ramifications.}

Coiffer:

\begin{itemize}
\item
  Coiffer qqn = to do sb's hair
\item
  Se coiffer, se peigner~= to do one's own hair~
\item
  {Faire les/ses cheveux}~
\item
  (Mettre de l'ordre à) (arranger)~ses cheveux = fix one's own hair
\item
  peigne = comb
\item
  peignoir = dressing gown
\end{itemize}

école:

monsieur/maîtresse (primary school) = teacher

bureau = desk

craie = chalk

salle de classe = classroom

devoir (de) maison = homework

contrôle = exam

{matière (d'enseignement), sujet = matter, material,~subject}

{Matériel scolaire = school equipment~}

{Réviser = to review (US), revise (BR), revisit}

{école primaire = primary school}

{Université = university, college~}

{Collège (m.) = high/middle school}

{prendre des notes = to take notes}

{Apprendre vs enseigner:}

\begin{itemize}
\item
  {Apprendre +~qqch +~à +~qqn: to teach, but if direct or indirect
  object is missing it means learn or study, and in this case~if we want
  to mean to teach enseigner should be used}
\end{itemize}

{taille-crayon, aiguisoir = pencil sharpener}

{feuille, papier = paper}

{chimie = chemistry}

{algèbre = algebra}

{travailler} {EN}{~}{binôme}{/groupe}{~}{= To work in pairs/groups}

{un projet} {DE} {groupe = a group project}

{first "Prendre un cours" then "Suivre le cours"}

\begin{itemize}
\item
  {Suivre un cours = Take a class (The French use "suivre" when you use
  "to take".)}
\end{itemize}

{au premier rang = in the front row}

{Réussir:}

\begin{itemize}
\item
  {avoir du succès = to succeed,~ to be successful, pass}
\item
  {Réussir l'examen = to pass the exam}
\item
  {Succès, réussite~}
\end{itemize}

{Distribuer = to hand out}

moyens de transport = means of transport:

\begin{itemize}
\item
  horaires = timetables
\item
  navire = vessel, ship
\item
  téléphérique = cable car/way
\item
  monter en = to get on
\item
  descendre de = to get off
\end{itemize}

c'est clair que qqch est ... =~qqch est clairement ...

pulvériser = to spray

Avenir, futur = future~

shampoing, shampooing = shampoo

mâcher = chew

proie = prey

chantilly = Sweet whipped cream

dompter = to tame

Occasions spéciales (anniversaire, mariage, ...)

sale:

en vente = for sale (at a normal price)

en promotion, au rabais (pejorative) = on sale/promotion/discount (at a
discounted/reduced price)

Rabais = discount~

\begin{itemize}
\item
  Je vous fais un rabais = I'll give you a discount~
\end{itemize}

faire grève = to strike, go on strike

rejeter = to reject

approuver = to approve

insupportable = unbearable

meurtrier = murderer

brouiller, Mêler en agitant, en dérangeant. Rendre trouble. Troubler par
brouillage.

épave = wreck

Déchaînée = Unleashed

child:

\begin{itemize}
\item
  enfant gâté = spoiled child
\item
  gamins = kids
\item
  enfants = children
\end{itemize}

béatitude = bliss

paradis terrestre = Heaven on Earth

augmentation de LA qualité = increase in quality~

la baisse du nombre = the decline IN the number

de pire en pire = Worse and worse

La baisse du nombre de = The decline/decrease in the number of

Se délaver = wash away

\begin{itemize}
\item
  S'est délavé = has faded
\end{itemize}

Le temps file = Time flies

j'ai vu de mes yeux = I saw with my eyes

bégaiement = stuttering

\begin{itemize}
\item
  bégaie = stutters
\end{itemize}

s'envoler = fly away

égarer, Mettre hors du bon chemin. = mislead, misplace, mislay

animal:

\begin{itemize}
\item
  moineau = sparrow
\end{itemize}

Fêter = celebrate~

Fête = celebration~

habituellement, d'habitude~= habitually, usually

Habituelle, coutumier (formelle), de d'habitude~= usual~

\begin{itemize}
\item
  Recette (habituelle)(de d'habitude) = usual recipe~
\end{itemize}

Surprendre = surprise~

Film =~Movie:

\begin{itemize}
\item
  Film dramatique = drama
\item
  Film romantique = romance~
\item
  Film d'action = action movie~
\item
  Film d'horreur = horror~
\item
  Film de science~fiction =~science fiction~
\item
  Dessin animé = cartoon~
\item
  Fantastique = fantasy~
\item
  Un film basé sur des faits réels = a film based on true events~
\item
  {je suis intrigué =} {I am intrigued}
\item
  {L'intrigue = plot (movie/book)}
\item
  {Le personnage (principal, secondaire) = (main, secondary) character~}
\item
  {Être situé dans = to be set in}
\item
  {Suite = sequel}
\item
  {Passionnant = exciting, fascinating~}
\item
  {Avoir une passion pour = to~ have a passion for~}
\item
  {Ennuyant = boring}
\item
  Aventureux(se) = adventurous~
\item
  Scénario = screenplay, filmscript, script
\item
  Tourner, filmer = to film
\end{itemize}

Peu probable = unlikely~

Avoir du mal à~

Être d'accord = to agree~

Être satisfait de = to be satisfied with

La cabine(fr)/ le salon(ca) d'essayage = the fitting/changing room~

J'ai oublié = I left/forgot~

Dans deux mois = in two months time

Donner envie de = to make you want~

Offrir un grand choix de = to offer a big selection~

Inside = dedans,~ à l'intérieur~

\href{https://www.larousse.fr/dictionnaires/francais/sortir/73545}{sortir}

will leave the nest:

\begin{itemize}
\item
  {Sortiront} du nid: sounds like they {could get back in right after}.
\item
  {Partiront} du nid: could work, it's just not as idiomatic as Duo's
  sentence.
\item
  {Quitteront} le nid: is used as a kind of fixed expression meaning
  {becoming an adult}.
\end{itemize}

{(sortir/partir DE) (quitter)} {quelque part = To leave (come out of)
somewhere:}

\begin{itemize}
\item
  {quitter ma/la maison} {(needs direct object}{) (}{quitter} {chez
  moi}{)}
\item
  {Sortir/partir de chez moi ==~}{Sortir/partir de~} {la} {ma} {maison}
\item
  {Sortir/partir du travail = to leave work~}
\end{itemize}

{music:}

\begin{itemize}
\item
  {jouer d'un instrument = to play an instrument}
\item
  {battrie = drums}
\item
  {se produire = to perform}
\item
  {groupe = band}
\item
  {instrument de musique = music instrument}
\end{itemize}

{fracas, Bruit violent = crash}

{sibling = frère et sœur}

{savoureux = tasty}

{loin, éloigné = far, distant}

{Tempête De Neige = blizzard, Snow storm}

{se connecter à Internet = to connect to the Internet}

{... in the ...:}

\begin{itemize}
\item
  {en~}{ville = in the city}
\item
  {le matin}{= in the morning}
\item
  {à la compagne = in the countryside}{~}
\end{itemize}

{(60 eruos)~}{au total = in total}

{J'ai cours = I have class}

{assez adj. = quite/fairly adj.}

{séjourner, rester = to stay}

{tenir au courant = keep informed/posted}

\begin{itemize}
\item
  {Nous te tiendrons au courant}
\end{itemize}

{Métier, emploi, occupation, boulot~}{(familier/informal, casual),
taf~}{(familier)}{:}

{Travail (de rêve)(de mes rêves)}

{Trouver un travail stable~}

{Dans quoi} {travailles-tu? Quel est ton travail? Je travaille dans ...}

{Informaticien : computer specialist}

{Chargé de clientèle : customer service specialist~}

{Fonctionnaire : public service worker}

{charpentier = Carpenter}

{comptable = accountant}

{fermier, agriculteur = farmer}

agent de bord,~hôtesse de l'air = flight attendant

steward = steward

\begin{itemize}
\item
  Maître d'hôtel ou garçon de service, à bord d'un paquebot.
\item
  Membre (homme) du personnel de cabine d'un avion.
\end{itemize}

Heures de travail = working hours~

Avion:

{le service à bord = on-board/inflight service}

Bon vol = have a good flight~

Bon voyage = have a good trip~

Un bagage à main = a hand luggage~

Un sac à main = a handbag~

to check in:

\begin{itemize}
\item
  hotel:~S'enregistrer
\item
  airport:~Enregistrer
\item
  ~J'ai fait l'enregistrement en ligne = I did the online check-in
\item
  Le guichet d'enregistrement, de check-in = check-in counter
\end{itemize}

s'en aller = to checkout, go away, leave

La carte d'embarquement = boarding pass

La porte d'embarquement = boarding/departure gate

voyager dans d'autre pays

au bord de = on the edge/verge of

papillon:

\begin{itemize}
\item
  papillon = butterfly
\item
  nœud papillon = bow tie
\end{itemize}

Que vous nous aimiez aussi~

entrevu, entretien d'embauche = job interview

\begin{itemize}
\item
  embaucher = to hire
\end{itemize}

un gars = a guy

amour:

\begin{itemize}
\item
  se fiancer = to get engaged
\item
  Engagé = Engaged
\item
  séduire, flirter = to flirt
\item
  vie sentimentale = sentimental/love/emotional life
\item
  embrasser = to kiss
\item
  {enlacer} = to embrace, hug
\item
  (petit) copain, petit ami, compagnon = boyfriend, partner
\item
  (petite) copine, petite amie, partenaire = girlfriend, partner
\item
  sortir en amoureux = go out as a couple, go on a date
\item
  sortir (avec) = to date
\item
  rompre = to break up
\item
  demander {le} divorce = ask for a divorce
\item
  tomber amoureux(se) = to fall in love
\end{itemize}

disponible/dispo/libre:

\begin{itemize}
\item
  être disponible = to be available
\item
  la première date disponible
\end{itemize}

esthéticien = beautician

rendez-vous~à/au/en/pour:

rendez-vous d'affaires = buisiness meeting

prendre~rendez-vous = to make/take an appointment~

\begin{itemize}
\item
  {prendre~rendez-vous POUR lundi}
\item
  {prendre~rendez-vous CHEZ/AVEC le dentiste}
\end{itemize}

{déplacer} le rendez-vous = to reschedule/move the appointment

rendez-vous galant/romantique/d'amour, rancard = date

{changer} l'horaire = change the schedule/time

Déplacer :

Dans le temps:

\begin{itemize}
\item
  Deplacer le Rdv
\end{itemize}

Dans le lieu:

\begin{itemize}
\item
  Déplacer la valise
\end{itemize}

à pied, en marchant = on foot

auberge = hostel

volontier = gladly (my pleasure)

Jusque là = up to now, till then,~ thus far, till now, until there

Se retrouver/réunir~avec des amis = to meet up with friends~

Se réunir = to assemble~

Se voir = to see each other~

Je n'ai rien de prévu = I have nothing planned, I have no plans~

Carrément~

Terrain = field~

Aidez vs au secours (à l'aide):

\begin{itemize}
\item
  Aidez : imperative, command, assist/help sb to do sth
\item
  Au secours : plea
\end{itemize}

En/y~Penser = to think about it:~

En penser (de): to Express an opinion on sth; to think of:

\begin{itemize}
\item
  Le mariage, qu'est-ce que tu en penses?
\end{itemize}

Y penser (à): to think about sth, to look to the future:

\begin{itemize}
\item
  Le mariage, tu y penses?
\end{itemize}

Être content/heureux/mécontent/malheureux DE = to be happy with

Produit en France = made in France~

Boulangerie, pâtisserie (pastry) = bakery~

Fait-main = handmade~

Aire conditionné = air conditioning~

Garner

changer:

changer: To bring about change to sth:

\begin{itemize}
\item
  ~Je veux changer LE pays: I want to change the country (for the
  better, perhaps)
\end{itemize}

changer de: switch exchange sth:

\begin{itemize}
\item
  ~Je veux changer DE pays:~I want to change countries (move somewhere
  else)
\end{itemize}

le livre s'intitule = the book is called

donner (avoir vue) sur = to have a view of, to look out onto

\begin{itemize}
\item
  notre chambre (a vue) (donne sur) sur l'autoroute
\end{itemize}

comédien, humoriste = comedian

connaître = to be familiar/acquainted with (indicates acquaintance)

savoir = to know how to (indicates knowledge of a fact or an action)

lit:

\begin{itemize}
\item
  {sur} le lit = {on} the bed
\item
  {dans} le lit = {in} the bed
\end{itemize}

aurais pu faire = could have done, would have been able to do

\begin{itemize}
\item
  aurais quand même pu faire = could {still} have done
\end{itemize}

s'envelopper dans = wrap oneself in

éblouissant = dazzling

To want:

désirer (more polite, elegant)= to desire, want:

\begin{itemize}
\item
  Dans magasin : qu'est-ce que vous désirez ?vous désirez ?
\end{itemize}

Vouloir~

\begin{itemize}
\item
  Dans magasin: qu'est-ce que vous voulez ? So the shopkeeper is either
  afraid of or angry with you~
\end{itemize}

souhaiter = to wish (want, more polite)

Ça te/vous dit/dirait~de/que = to ask whether sb feels like doing sth:

\begin{itemize}
\item
  Ça te dirait (qu'on aille) (d'aller)~au resto
\end{itemize}

trouble-fête = Spoilsport. a person who behaves in a way that spoils
others' pleasure, especially by not joining in an activity.

J'ai fait un cauchemar = I had a nightmare~(at sleep)

rêve:~

\begin{itemize}
\item
  J'ai fait un rêve = I had a dream (at sleep)
\item
  J'ai un~rêve =~I have a dream (to accomplish in life, having
  expectations)
\end{itemize}

à l'aide, au secours = help

cru = raw

sauterelle = Grasshopper

écureuil = squirrel

herbe = grass

pelouse = lawn

tondre = to mow

véranda (f.) = veranda. A veranda or verandah is a roofed, open-air
gallery or porch, attached to the outside of a building. A veranda is
often partly enclosed by a railing and frequently extends across the
front and sides of the structure.

\begin{itemize}
\item
  {dans} la~véranda
\end{itemize}

terrasse = terrace

\begin{itemize}
\item
  {sur} la~terrasse
\end{itemize}

gymnase (m.), gymnasium = gymnasium

Gym, gymnastique (f.)~= gym

Muscu, musculation, haltérophilie~= weight lifting~

Aérobie = aerobics~

Entraînement = training

Étirement = stretching~

Entraîneur personnel~= personal trainer

Abonnement = membership~

Soulever = to raise, lift

Vestiaire = locker room~

planifier = to plan

tirer = to throw, kick, shoot

\begin{itemize}
\item
  il tire ({dans} le) au ballon = he shoots/kicks the ball
\end{itemize}

attacher = to attach, tie, fasten

s'entraîner = to practice

Pilates (m., singl.) = pilates~

partie (f.) = match

course (f.) = race

filet (m.) = the net

costume (m.) = suit

Robe de mariée = wedding dress~

Bourse (f.)~(capital b) = stock exchange~

bourse (little b) = scholarship, grant

(faire un) séjour linguistique: (take (go on)) a language trip

\begin{itemize}
\item
  séjour linguistique = language stay/study/journey,~language study trip
  ((international) student exchange),~language study~exchange
\item
  séjour linguistique à l'etranger = language study abroad
\item
  séjour d'étude de la langue = language study trip
\item
  programme d'échange étudiant = student exchange program
\item
  échange scolaire = school exchange
\item
  language exchange
\end{itemize}

famille d'accueil = host family

À l'envers = backwards~

Épeler~= to spell~

Échouer à = to fail (a/an test/exam, IN a competition)

divertissant = entertaining

maison:

armoire = wardrobe, closet

placard = cupboard

sous-sol = basement

toit = roof

plafond = ceiling

Des chiottes, toilettes~

sink:

\begin{itemize}
\item
  kitchen: évier~
\item
  bathroom,~restroom/toilet: lavabo (no need to add de la salle de bain)
\end{itemize}

la cuvette des toilettes = the toilet bowl

balayer = sweep

après shampoing = conditioner

prendre une douche, se doucher = to take a shower, shower

souper,~dîner (m.) = supper

(prendre le) dîner = to have dinner, to dine

prendre le petit-déjeuner = to have breakfast

gazinière = gas cooker, stove

Saladier = salad/mixing bowl~

Verser = to pour~

Pâte (f.) = dough

Poêle (f.) (à frire) = frying pan

Se colorer = to turn (color)

Laisser cuire = to let cook

Louche (f.) = ladle~

Un puits = a well

Une pincée de sel = a pinch of salt

la tasse à mesurer = the measuring cup

épicé,~pimenté,~piquant = spicy, spiced

amère (f.), amer (m.) = bitter

peler = to peel

lave-toi les mains = wash your hands

passe-moi le sel =~pass me the salt

hiatus = hiatus

Camarade/copain~de classe = classmate

Chose (f.)~vs truc (m.)

\begin{itemize}
\item
  Chose standard, truc familier/colloquial~
\item
  Chose = thing, truc = stuff
\end{itemize}

Mère au foyer = stay-at-home mother~

Stronger (child): je veux une glace~

Adult: j'ai envie d'une glace~

supermarché (m.) = supermarket

sac plastique

sac en/de papier = paper bag

allée (m.) = aisle

\begin{itemize}
\item
  dans l'allée = in the aisle/driveway
\end{itemize}

{Caisse (f) = box, (cash) register/till}

\begin{itemize}
\item
  {Une caisse de vin = a case/crate~of wine (wine is in the case)}
\item
  {Une caisse à vin = a wine case/crate (wine could be in the case or
  not)}
\end{itemize}

{chariot = cart, trolley}

{produits laitiers = dairy products}

{sucreries = sweets}

{Confiseries = confectionery~}

{j'aimerais acheter du X =~}{I would like to buy X}

{où se trouve-t-il = , where is it}

{où puis-je en trouver}

\begin{itemize}
\item
  {il se trouve dans ...}
\end{itemize}

{auriez-vous du X}

se rendre, aller:

\begin{itemize}
\item
  comment t'y rends-tu ? = how do you get there?
\end{itemize}

arbre généalogique = family tree:

\begin{itemize}
\item
  du côté de ma mère = on my mother side
\item
  arrière grand-mère = great grandmother
\item
  arrière petit fils = great grandson
\item
  proches = relatives
\end{itemize}

personnalité, traits de caractère = character traits:

talentueux = talented

fainéant,~paresseux = lazy

impulsif,~spontané =~impulsive

cynique = cynical

prudent =prudent, careful, cautious

sensible = sensitive

vaillant = valiant, brave

\begin{itemize}
\item
  Plein de {bravoure}, de courage, de valeur pour se battre, pour le
  travail, etc.
\item
  Qui {vaut} qqch.
\end{itemize}

aimable = friendly, kind

sympathique, sympa = friendly, nice

extraverti = extrovert, outgoing

gracieux = gracious

à ta place = in your place, if I were you/(in your shoes)

vacance(s):

une vacance (singl.): There is a place left in a company (a vacancy)

des vacances (pl.): holidays

\begin{itemize}
\item
  C'est les vacances.
\item
  Ce sont les vacances.~
\end{itemize}

To enjoy:

profiter de = to enjoy, make the most of, benefit from

\begin{itemize}
\item
  Profite de tes vacances ! = Enjoy your holidays !
\end{itemize}

S'amuser = to have fun, enjoy

\begin{itemize}
\item
  Amuse-toi beaucoup = have a lot of fun
\end{itemize}

s'éclater (s'amuser de façon très vive) = to have fun, to have a great
time

Amuser = to entertain/amuse others

\begin{itemize}
\item
  {Parc d'attractions = amusement park~}
\end{itemize}

{Amusant = fun}

{Se divertir = to~entertain~}

Réfléchir vs penser:

\begin{itemize}
\item
  ne pense pas au travail. = don't think about work.
\item
  Tu es trop guidé par tes émotions. Réfléchis !
\end{itemize}

caractéristiques physiques = physical characteristics:

\begin{itemize}
\item
  à quoi ressemble-t-il = what does he look like,~what is he like
\item
  cheveux roux (f.: rousse) = red hair
\item
  cheveux bouclés = curly hair
\item
  cheveux raides = straight hair
\item
  brun(e) = brunette
\item
  beau gosse = handsome
\item
  belle gosse = good-looking girl
\item
  mince = thin
\item
  maigre = skinny
\item
  Peau Claire = fair skin
\end{itemize}

Similar:

\begin{itemize}
\item
  Poignet (m.) = wrist
\item
  Poignée (f.) de porte~= doorknob, door handle~
\item
  Cheveux = hair
\item
  Chevaux = horses
\end{itemize}

Pourtant, cependant, toutefois = yet

comment te sens-tu? = how do you feel

aise (n.), aisé (adj.):

\begin{itemize}
\item
  malaisé, mal à l'aise, embarrassé = uneasy/uncomfortable, embarrassed
\item
  à l'aise = comfortable
\item
  IncoNfortable = Uncomfortable
\end{itemize}

se sentir:

\begin{itemize}
\item
  frustré = frustrated
\item
  confus(e) = confused
\end{itemize}

Pouvoir,~Être capable de, arriver à, parvenir à = to be able to~

attraper = to grab, catch

à ce moment là = at this/that moment

prendre au sérieux = take seriously

à/de nouveau = again

bibliothèque = library

\begin{itemize}
\item
  emprunter un livre = to borrow (check out) a book
\item
  pénalités de retard = late penalties/fees
\item
  la carte est suspendue = the card is suspended
\item
  bibliothécaire = librarian
\item
  rapporter ses livres à~temps= bring back his books on time
\end{itemize}

Retard = delay

Attardé = retarded~

assiete = plate, dish

bol = bowl

À part de = Apart from

intervenir = interject, say (something) abruptly, especially as an aside
or interruption.

apprendre par coeur = learn by heart

transpirer = to sweat

Nourriture:~féculent,~protéine,~légume, fruit, boisson,~édulcorant

fécule = starch

édulcorant = sweetener

puis-je avoir = can you bring me (can/may I have) (to ask a favour)

appeler le serveur~~= to call over the waiter

coup de coude = nudge (accidentally nudge sb as you pass by)

politesse (f.) = politeness, manners:

les excuses:

\begin{itemize}
\item
  pardon: to apologize for small things (bumping into sb)
\item
  Je suis desolé(e) = feel guilty
\item
  excuse(z)-moi = to get sb's attention (waiter), or interrupting sb
\item
  navré = (I am) sorry (in sad situations), can be interchangeable with
  pardon
\item
  mes excuses = my excuses/apologies, can be interchangeable with pardon
\end{itemize}

changer de {carrière} = to change {careerS},~to change his {career~}

pour y aller = to get/go there

savez-vous où se trouve le zoo = do you know where (the way to) the zoo
is

ensemble ou séparément = together or separately

avec plaisir = my pleasure

un morceau de = a piece of

je trouve = I think so

\begin{itemize}
\item
  je trouve aussi = I think so too
\end{itemize}

près de la fenêtre = near/by the window

grandes-personnes = big people, adults

mâcher = to chew

avaler = to swallow

faire l'expérience~~= do the experiment

des tas de = A lot of

on est {égaré}, nous sommes perdus, on est perdus = we are lost

égaré = disoriented

proie = prey

fauve (m.) = Wildcat, jungle cat, wild beast

le train-train quotidien = everyday routine

pressentiment = foreboding

douceur (f.) = sweetness

vous éviter de = save you from

rosier = rosebush

accueillir = to welcome, host

inhabité = uninhabited

soudain = suddenly, all of a sudden

sur le chemin du retour = on the way back

foire (f.) = fair

dans les environs = in the (surrounding) area

bête (f.) = beast, stupid

aîné = elder

rugir = to roar

la demande en mariage = to ask her to marry

il était une fois = once upon a time

dans un murmure (m.) = in a whisper

supplier = to beg

en pleurs = in tears

prévenance (f.) = thoughtfulness

éclater = to burst

fée = fairy

conte (m.) = tale

Se rassembler = to get together~

Rendre hommage = to pay hommage~

Nous rappelle = reminds us~

Dette = debt

Au nom de = in the name of

Le lendemain,~ le jour d'après/suivant~

Rentrer (qqch):

\begin{itemize}
\item
  To go/come back/in home
\item
  To get/bring/take (sth) back~inside
\item
  {Rentrer/retourner chez soi (à la maison) = to go home}
\item
  {Rentrer dans = to fit to/into}
\end{itemize}

{endroit = place}

{région (f.), zone (f.), l'aire =~}{area, zone sector}

{quartier = neighbourhood, district}

Se garer sur une~Place (un endroit) interdit(e)= to park on a prohibited
spot

Se garer au bon endroit/lieu~

Garer:

\begin{itemize}
\item
  J'ai garé la voiture dans le garage~
\item
  Je me suis garé dans le garage~
\end{itemize}

Aide, assistance = help

Ça/il/elle va bien avec = it goes well with~

Quand + futur simple = when + simple present:

\begin{itemize}
\item
  Ne panique pas quand tes beaux-parents arriveront = don't panic when
  your parents-in-law arrive
\end{itemize}

Faire une remarque à~ qqn~= to make a point to sb

Sombre, ténébreuse = dark

Rue:

\begin{itemize}
\item
  Dans la rue: at the side of the street, on the pavement~
\item
  Sur la rue: in the path of the traffic~
\item
  In(Fr/Br)/On(Am) the street~
\item
  On the street (in Br): homeless~
\end{itemize}

Walk:

\begin{itemize}
\item
  Faire une promenade, se promener = to take a walk,~ take walks
\item
  Promener = to walk (pets)
\item
  Marcher dans qqch = to walk along/down sth
\item
  Marcher sur = to step on, tread (past: trod)
\item
  {Marcher = step, walk,~}{to trek (go on a long arduous journey,
  typically on foot.}
\item
  {En chemin = on the way~}
\end{itemize}

Plus:

\begin{itemize}
\item
  S'il ne pleut plus: if it doesn't rain anymore, if it no longer rains
\item
  S'il ne pleut pas plus: if it doesn't rain more~
\end{itemize}

Au milieu de = in the middle/midst of, amid

baby:

\begin{itemize}
\item
  Human: bébé
\item
  animal: petit
\end{itemize}

grow:

\begin{itemize}
\item
  plantes: pousser, croître, cultiver
\item
  Human, animal: grandir
\end{itemize}

grand:

grand + noun = big, great

noun + grand = tall

\begin{itemize}
\item
  under taller trees = sous des arbres plus grands
\end{itemize}

rayure = stripe

nid = nest

sommelier = sommelier, wine waiter

souliers, Chaussure épaisse, qui couvre bien le pied. = shoes

C'est à un kilomètre = it's one kilometer away.

dépanneur, Professionnel(le) (mécanicien, électricien, etc.) chargé(e)
de dépanner. Voiture de dépannage qui peut remorquer les automobiles en
panne.= convenience store, repairman, serviceman

vignoble = vineyard

une appellation contrôlée = a controlled wine region

dégustation de vin = wine tasting

bon visite = good visit, have a great visit

peinture à l'huile = oil painting

aquarelle = watercolor

exposition = exhibit

essayer = to try, try on

achat = purchase

\begin{itemize}
\item
  retourner un achat = return a purchase
\end{itemize}

hein ? = eh?

politique = policy, politics

rien de blanc = nothing white

jeux de société = board games

jouer aux cartes = to play cards

la batterie = drums

faire du sport = to play/do sports, exercise

faire de l'exercice = to do exercise

pondre = spawn

Climb

\begin{itemize}
\item
  monter/grimper~dans l'arbre = climb (up/in) the tree
\item
  Grimper sur le toit = to climb on the roof
\end{itemize}

terre = soil, Earth, dirt

care:

prendre bien soin de, s'occuper bien de = to take good care of

S'occuper des enfants = take an eye of~them, keep them busy, look after
them

Prendre soin des enfants = babysitting, nursing, take care of their
needs~

Il n'en prend pas bien soin~

On s'en occupe :~

\begin{itemize}
\item
  To provide care for a living being
\item
  To get sth done, don't worry~
\end{itemize}

se soucier/préoccuper/inquiéter de, être inquiet(s) de, s'inquiéter
pour~= to worry/care about (sth: de)(sb: pour), to be worried about

\begin{itemize}
\item
  Tu t'inquiètes pour moi (intransitive)~= you worry (yourself)~about
  me~
\item
  Tu m'inquiètes (transitive) = you worry~me, you make me worry~
\item
  Pourquoi tu t'inquiètes pour moi = don't worry plz
\item
  Pourquoi tu te préoccupes de moi = mind your own business~
\item
  Se préoccuper de (to worry about sth)~is used for abstract things
  (weather, situation, problem,...) in contrast to s'inquiéter
\end{itemize}

attentionné, Qui est plein d'attentions pour qqn = attentive, caring

I don't care about (top (less colloquial) -\textgreater{} buttom (more
colloquial)):

\begin{itemize}
\item
  s'en moquer, se moquer de,~s'en moquer de (en is not necessary, it is
  for emphasis)
\item
  s'en ficher,~se ficher de,~s'en ficher de~(en is not necessary, it is
  for emphasis)
\item
  s'en foutre,~se foutre de,~s'en foutre de (en is not necessary, it is
  for emphasis)
\item
  ballec de
\end{itemize}

bâton de marche = walking stick

noyau
{[}\href{https://stephaniehuesler.com/2020/01/04/seeds-pits-pips-or-stones/\#:~:text=Pits\%2520are\%2520found\%2520in\%2520fruits\%2C\%252C\%2520avocadoes\%252C\%2520olives\%2520and\%2520dates.}{stephaniehuesler}{]}:

\begin{itemize}
\item
  SEED: A fertilized and ripened ovule*, containing an embryonic plant.
  {[}*the structure in a plant that develops into a seed after
  fertilization.{]}
\item
  PIT is a seed, stone or pip inside a fruit, or a shell in a drupe
  (such as a peach) containing a seed.
\item
  PIP is a British term for a seed inside certain fleshy fruits (compare
  stone/pit), such as a peach, orange, or apple!
\item
  STONE is the central part of some fruits, particularly drupes;
  consisting of the seed and a hard endocarp layer.
\item
  If I had to put it in layman's terms, I'd say it like this: The seed
  contains the embryo; the pit/pip/stone protects the seed until it's
  ready to sprout (and only certain types of fruits have pits); pits are
  usually singular in a fruit, while there may be one or more seeds.
  Pits are found in fruits like cherries, mangoes, peaches, plums,
  avocadoes, olives and dates. Seeds are found in fruits like apples,
  oranges, and bananas (the variety of bananas usually sold in stores
  usually have sterile seeds -- what we might call ``seedless'').
\end{itemize}

démentir = deny, refute

s'emparer = seize

le papier/document d'identité = identification

vers = toward, around

Les voici = here they are

madame = ma'am

est compris = is included

\begin{itemize}
\item
  est-ce que le petit-déjeuner est compris = is the breakfast included
\end{itemize}

tant/autant de variétés = so many varieties

24 heures sur 24 = 24/7, 24 hours a day

Chambre à deux lits = room with 2 beds

Chambre au grand lit = room with a big/double~bed

Chambre à une salle de bain privée = room with a private bathroom~

Installer:

= to set up: if assembly is required:

\begin{itemize}
\item
  help me set up this flat pack bookcase
\end{itemize}

= to set: for things that are ready (no assembly is needed)~and just
need programming/tuning/placement:

\begin{itemize}
\item
  Set the clock~
\item
  Set the oven temperature~
\item
  Set that on the table~
\end{itemize}

sous peu = soon, shortly

Prendre = to have (food and drink), partake (Br.~over-formal)

Café = café, coffee~

les verts = the green ones

Chaîne (f.) = channel~

Abonne-moi = follow me (social media)

Suivez-moi =~follow me (resto)

Je vais t'abonner = I am going to follow you~

se présenter, présenter = to present/introduce oneself

Je te présente Sarah = I introduce Sarah to you~

pour cela:

\begin{itemize}
\item
  C'est ... = This is ...
\item
  Je suis ... = I am ...
\item
  Je m'appelle ... = My name is...
\end{itemize}

salutations = greetings:

\begin{itemize}
\item
  Salut, ciao = hi, bye
\item
  Bonjour = hello, good morning/afternoon
\item
  Bonsoir = hello, good evening~
\item
  D'où êtes-vous ? D'où venez-vous ?
\item
  Je suis/viens~de (ville)
\item
  Comme si, comme ça = so, so
\end{itemize}

Pupitre (m), table = desk, music stand

C'est génial/merveilleux/sensationnel/fantastique = it's great~

Quoi (qu'y a-t-il) de neuf = what's new/up, sup

Coudre = to sew (sewed, sewn)

Du fil et des aiguilles = needles and thread

Réparer: repair, fix, mend/darn (for socks/clothes. Darn: to fix a hole
by weaving the thread or yarn)

Contrarié = upset

Être AU sec~

La semaine passée(dernière)/prochaine~= last/next~week~

La veille de = the day before~

À l'époque = at the time~

On s'inquiète de, on est inquiets de = we are worried about~

préoccupé = worried, concerned

Chemin de fer = railway~

Laitue (f.) = lettuce~

Bleuet = blueberries~

Croquer = Munch~

Croustilles = potato chips, crackers (biscuits salés)

Jurer = to swear~

Discours (singl.) = speech~

Hocher la tête = to nod (one's head), to give a nod (of the/one's head)

Secouer la tête = to shake one's head~

Gros mot = swear/curse/bad/cuss word, profanity~

Dire des gros mots = to swear~

À tout à l'heure, à plus tard = see you later~

À la prochaine = see you next time~

À bientôt = see you soon~

Couper la parole à qqn = to cut sb off

\begin{itemize}
\item
  me couper la parole
\end{itemize}

TierS (sing.) = one third, third party~

Les baisses d'impôts = (the) decreases in taxes, tax cuts

Une baisse de la qualité = a decrease IN {of} quality~

Ne faire plus aucun doute = to leave no doubt~

Se servir de, s'en servir~

se faire faire qqch = to have/get sth done

À un moment donné = At one point

Progrès :

\begin{itemize}
\item
  (Sing.) usually a concept showing improvement in science and
  technology: grâce au progrès on peut soigner cette maladie.~
\item
  (Pl.) little steps one makes while getting better at sth: je suis nul
  mais je vais faire des progrès~
\end{itemize}

Pour en revenir à = getting back to~

Conclure~

En conclure = to came to a conclusion about/from sth without mentioning
what the conclusion is actually based on: to conclude about/from
that/it, come to the understanding, figure out, take it

En:

\begin{itemize}
\item
  De ce que tu as dit
\item
  De ce que je vois~
\item
  De ce que j'ai compris~
\end{itemize}

Conclure = (without en)

Without mentioning the base of conclusion: to~end sth: to conclude, end,
finish

\begin{itemize}
\item
  to conclude a meeting~
\item
  Conclure le rapport~
\end{itemize}

With~mentioning the base of conclusion: to conclude:

\begin{itemize}
\item
  Conclure de qqch que...
\item
  Je conclus de votre rapport que ...
\end{itemize}

On ne~peut rien conclure/finir/terminer~= we cannot finish/end anything~

On ne peut rien en conclure = we cannot conclude anything about/from it

Pour conclure: in conclusion, to conclude/ wrap up.~

\begin{itemize}
\item
  Pour conclure, nous n'avons rien fait~
\end{itemize}

À conclure: left~to conclude~

\begin{itemize}
\item
  Il nous reste à conclure cela~
\end{itemize}

Par conclure: by concluding.

\begin{itemize}
\item
  On a fini par conclure que nous n'avons rien fait~
\end{itemize}

Sans conclure: without concluding~

\begin{itemize}
\item
  Nous avons fait le rapport sans conclure~
\end{itemize}

déceler = detect

bonne sœur = nun

des chose futiles = futile things

traîtresse = traitor

antipathique = unsympathetic

extrait (m), Passage tiré d'un texte = extract, excerpt

Calamar = squid~

Des coquillages = seashells~

Oursin = sea urchin~

Hippocampe = seahorse~

Crabe (m) = crab

Poulpe = octopus~

Le paréo = sarong~

Les pantoufles = slippers~

Les tongs/tongues
{[}\href{https://soreez.com/blogs/blog/tong-ou-tongue-quelle-est-la-bonne-orthographe}{soreez}{]}
= flip-flops~

Palmes = flippers~

Chaise~longue = lounge chair~

({dans} le) Parasol = beach umbrella,~ parasol~

Vinaigre (m) = vinegar~

Épices = spices~

Épicerie = grocery store~

Épicier = grocer

Crêpe = pancake~

Charcuterie = delicatessen~

Moutarde (f) = mustard~

Hat:

\begin{itemize}
\item
  Chapeau (summer)
\item
  Bonnet (winter)
\end{itemize}

Scarf:

\begin{itemize}
\item
  Foulard (thin, summer)
\item
  Écharpe (winter)
\end{itemize}

La combinaison = overall~

Peignoir (m) = bathrobe~

Les collants = tights

Les bas = stockings~

Ruisseaux = streams~

abandonner:

\begin{itemize}
\item
  Résigner = resign~
\item
  Give up
\end{itemize}

Creux = hollow, empty~

Le draps = the sheets~

Le draps de lit = bed linen (sheets, pillowcases, and duvet covers)

Tabouret = stool

Cuillère = spoon~

Casserole = pan

Cuisinière = cooker, cook

Marqueur = felt tip

Un flip-chart = a~flip chart

Un tableau blanc = a white board~

tableau = chalkboard (blackboard), painting

L'agrafeuse = the stapler~

Le bloc-notes = notepads, bloc-notes

Les classeurs = folders, binders

photocopieuse = the photocopy machine~

Un open space = an open space office~

Un bureau individuel = a private office~

Tousser = to cough

La grippe = a flu

Endocrinologue = endocrinologist

Pédiatre = pediatrician

Reçu = receipt~

Payer le reste = to pay the rest

bon marché = cheap, inexpensive

Foie (m) = liver

Café crème = cappuccino~

bill =~

\begin{itemize}
\item
  Addition (f.): At restaurant
\item
  facture (f.): At hotel (checking out)
\end{itemize}

Ça vous a plu? = did you like/enjoy it

Recommander = to recommend~

Commander = to order

Entrée = appetizer:

plats principaux~= entrées

En/comme/(pour l') entrée = as/for (a) starter

\begin{itemize}
\item
  Entrée is part of a sit down meal, while hors d'oeuvre is finger food,
  before the main and sit down~meal~
\end{itemize}

Un apéritif = ou familièrement un apéro, est une boisson, servie avant
le repas dans certaines cultures afin d'ouvrir l'appétit.

Digestif = Un digestif est une boisson, alcoolisée ou non, qui se prend
habituellement à la fin d'un repas, et qui est censée faciliter la
digestion et revigorer à la fin d'un long repas.

ranger = clean up, tidy

Chambre (à coucher) = bedroom~

lumineux = luminous

Méfie-toi de = be wary of~

un anneau dans/pour le nez = a nose ring

S'en rendre compte = to realize it:

\begin{itemize}
\item
  Ils s'en rendent compte = they realize it
\item
  Ils ne~s'en rendent pas~compte = they don't~realize it
\item
  Ils s'en sont rendus compte = they realized it
\item
  Ils ne~s'en sont pas~rendus compte = they didn't~realize it
\item
  Je regrette qu'ils ne s'en soient pas rendus compte : I regret that
  they didn't realize it
\end{itemize}

rédaction, exercice scolaire qui consiste à traiter par écrit un sujet
narratif. = essay, composition, writing

Fringues,~ vêtements:

\begin{itemize}
\item
  être (à) la bonne taille = to be the right size
\item
  (à) ma taille = in my size
\item
  Costard = suit
\end{itemize}

D'occasion = used

C'est dommage = it's a pity~

C'est une honte = it's a shame~~

Beaux parents = parents-in-law, in-laws (colloquial), outlaws
(colloquial)

Rester : to stay, remain~

Surtout : above all~

FAIRE~un tatouage/des blagues~

Chaque centimètre = {EACH~}EVERY~centimeter/inch:

\begin{itemize}
\item
  Each: separeate individuality
\item
  Every: totality
\item
  Each and every: to emphasize~
\end{itemize}

Éviter = to avoid, dodge

Peur:

\begin{itemize}
\item
  Ça fait peur = it's scary~
\item
  Qqch fait peur {À} qqn = sth scares sb
\item
  faire peur = to scare, make scared
\item
  {Avoir peur de faire = to be scared/afraid} {OF} {doing (not TO do)}
\item
  {être effrayé(e) = to be afraid/frightened}
\end{itemize}

Couvert DE = covered with/in

Poussière = dust~

Crier À qqn = to shout out to sb

Moisissure (f) = mold

Étincelle = spark

Warm:

S'échauffer = to warm up

To be warm:

\begin{itemize}
\item
  La chaussette est {réchauffante}.
\item
  Le thé est chaud~
\item
  Il fait chaud~
\end{itemize}

Terre = dirt

Sucré et acide, aigre-douce = sweet and sour~

Nid = nest

Arbuste (m.), ({dans} le) buisson = bush,~shrub

Guêpe (f.) = wasp

Caillou (m) = pebble~

Haut de gamme = top of the line, upscale~

Ignorer, ne pas connaître/savoir = not to know

Ruche (f) = beehive~

Ça sent LE/LA qqch = it smells like sth

Piquer = sting

lavande (f) = lavender~

indications = directions:

en~direction de = to

indique-moi la route = show me the way

angle, coin = corner

premier, tout d'abord, avant tout = first, first of all, above all

puis = then

pour finir =~Finally (used when saying a sequence of plans)

Station d'autobus, gare routière = bus station

Station de train, gare= train station

Côté :

Côté qqch = on the sth side:

Être côté couloir de = to be on the aisle side of

Côté couloir (adj.)

\begin{itemize}
\item
  Un siège côté couloir = an aisle seat~
\end{itemize}

Côté~fenêtre =~on the window side

À côté = nearby

À côté de = next to, beside

De côté = besides:

\begin{itemize}
\item
  laisser de côté = to~Leave besides
\end{itemize}

Du côté de = on the side of

\begin{itemize}
\item
  du côté de ma mère = on my mother side
\end{itemize}

De ce côté(-ci/là)~de~= on this(/that) side (here/there)~of:~

\begin{itemize}
\item
  De ce côté de l'avion~
\end{itemize}

à proximité = near, close by

\begin{itemize}
\item
  C'est {à proximité d}'un centre commercial = it's {near} a mall.
\end{itemize}

en face de = across from

à l'opposé (de) = at the opposite (of), on the opposite side

à la droite de = on the right hand side of

gauche:

\begin{itemize}
\item
  à la gauche de = on the left hand side of
\item
  prendre~à gauche = to take a left turn, to turn left
\item
  {sur} la gauche = on/to the left, on/to the left-hand side
\end{itemize}

devant = in front of

derrière = behind

entre = between

tourner = to turn

Traverser = to cross

continue tout droit = continue straight

{suivre cette route = to follow this road}

{par ici = over/down here}

{Par là = over there}

{Par là-bas = down there~}

{là bas = over there}

{là haut = up there}

Ex: Avant tout, prend la rue X, continue tout droit et traverse la rue
X, puis tourne (à droite au X) (dans la rue X)

centre

\begin{itemize}
\item
  centre de sports = sports center
\item
  centre commercial = mall.
\end{itemize}

Ramasser = to pick up, gather

Emporter, prendre avec soi et porter hors d'un lieu qqch ou qqn qui ne
se déplace pas par soi-même = to~carry, carry~ out, take, take out

Partir de chez soi = leave home

Poser qqch contre le mur = to put sth against the wall~

Sous menace de = under threat of~

Tremplin = springboard~

grignoter = munch on, nibble, snack

La peau du fruit = the skin/peel

se décider = to decide

au départ = at the beginning

la remise de diplôme = graduation

Canapé, divan, sofa = couch,~ sofa

fauteuille (m.), fauteuil = sofa

Laisser = to let, to leave

L'aspirateur = vacuum (cleaner)(US), hoover (UK, after an early brand)

Bricolage = do-it-yourself projects~

Être nul EN~qqch = to be bad AT sth

Le clou qui est DE mauvaise qualité = a poor-quality nail

Stagiaire = intern, trainee

Stage = internship, traineeship~

Fleur = flower, blossom (trees have blossoms not flowers)

germer = sprout

Récolte~= harvest~

Pourri = rotten~

Graine (f) = seed

Noyau (m) = pit

Semer = sow

Olivier = olive tree

Verger = orchard~

Potager = vegetable garden~

Se disputer = to argue~

Bien que = even though, although~

Tu fais exprès de faire = you are doing on purpose, you are trying to do

Ne pas accepter = to not accept/tolerate~

Regretter = being sorry,~ to regret~(about what you or others did),
déplorer (only about others. Stronger)

Un(e) drôle de + n~= a strange(odd, bizarre,
weird)/funny/incredible/terrific + n

Une drôle fille = a funny (comically) girl

\begin{itemize}
\item
  De drôles manières = bizarre means~
\item
  Des manières drôles = funny means~
\end{itemize}

Une drôle de fille = a funny (weird) (kind of) girl~

drôle, marrant(e) = funny

S'endormir = to fall asleep, go sleep

Il manque qqch1~DE/DANS qqch2 = sth1 is missing FROM sth2, there's a
shortage/shortfall of sth1 IN sth2

Ne fait plus = doesn't do anymore, no longer does~

Qqn rentre {dans qqch =~ sb fits in/into sth, sth fits sb}

\begin{itemize}
\item
  {Qqn rentre dedans}
\end{itemize}

{Se retrouver = to meet up}

{Se revoir = to see each other again~}

{Attention =? be careful, watch out}

{Faire attention = to pay attention~}

{Il faut du courage = it takes courage,~one needs courage~}

{Il va falloir = you/we are going to need, it is going to be necessary~}

{Mépriser = despise (to regard as unworthy of consideration;~to not
bother to think about)}

{panneau (m) = panel, sign}

{Anneau = ring (basic shape)}

{Bague = ring (finger ring, decorative,~ show social status)}

{Elle a mis du vernis à ongles = put on, applied}

{collant = tights}

{le vernis à ongles = nail polish/varnish (}{varnish applied to the
fingernails or toenails to color them or make them shiny.}{)}

{Sans raison = for no reason~}

{Un~Commentaire = a~comment}

{Une remarque = a~remark~}

\url{https://grammar.yourdictionary.com/vs/irony-vs-sarcasm-types-and-differences.html}

\url{https://education.toutcomment.com/article/quelle-est-la-difference-entre-l-ironie-et-le-sarcasme-exemples-et-explications-13848.html}

{Ironique (m) = ironic, ironical (it's just there. When the expected
outcome is the "opposite" of the actual outcome)}

{sarcastique (m)~=~}{sarcast}{ic (}{has a condescending tone meant to
embarrass or insult someone. Therefore, the negativity in sarcasm is the
clear difference.~}{peo}{ple make it happen. If sth you planned goes
wrong and sb says "well done", he is being sarcastic)}

{Sardonique (m) = sardonic (e.g.,~comedian's sneering jokes when
criticizing a specific group of people is called sardonic)}

{satire (f) = satire (making fun of people by imitating them in ways
that expose their flaws)}

{Le tatouage = tattoo~}

{Va-t'en, allez vous en = go away}

{Gêné = embarrassed~}

{Gênant = embarrassing~}

{Se poser = to land}

{S'envoler = to fly away, fly off}

{Voler = to fly, steal~}

{Reculer = to step back~}

{Le temps:}

{C'est une belle/mauvaise journée~}

{il fait 15 degrés = it's 15 degrees}

{l fait bon = La température, le temps est agréable.}

{quel temps fait-il,~ comment est le temps~= what weather is it, what is
the weather like}

{les points cardinaux = Cardinal points}

{il brouillasse, il y a du brouillard = there is fog, it is foggy}

{la grêle = hail}

{Il grêle = it hails~}

{le brouillard = fog}

{la brume = mist}

{tonnerre (m.) = thunder}

{éclair = lightning}

{le ciel couvert (/nuageu/gris)= the overcast sky}

{C'est nuageux, il~ y a des nuages~= it's cloudy}

{Il fait venteux, il y a du vent = it's cloudy~}

{Il y a du soleil = it is sunny}

{Prendre (le) (un peu de)~soleil = to get (some) sun}

{Au soleil = in the sun~}

{Doux = mild}

{canicule = heat wave}

{Pluvieux = rainy}

\begin{itemize}
\item
  {Il fait pluvieux~}
\end{itemize}

{le ciel} {dégagé} {= the clear sky}

{l'averse} {= the downpour}

{abordé = addressed, covered}

{klaxonner = to honk}

{tomber malade = to get/feel/become sick}

{d'une façon ou d'une autre = one way or another}

{ça s'arrête là = it stops there}

{comme si de rien n'était = Like nothing ever happened}

{effectuer le trajet = make the trip}

{allégresse, joie = gladness}

{apatride = stateless}

{aller-retour = round-trip, goings and comings, going back and forth}

{révoltant = revolting, shocking, appalling}

{révolter = revolt}

{indiscriminé = indiscriminate}

{commerçant = merchant, trader, tradesman, shopkeeper}

{rudoyer = to bully}

{rudoiement = rudeness}

{petitesse (f) = smallness}

{étendre = expand, extend, stretch out}

{étendre les jambes = stretch the legs}

{s'allonger = lie down, stretch out}

{sans bornes = without bounds, boundless, unbounded}

{ressentiment = resentment}

{être envahi de = to be invaded by}

{à l'avenant = accordingly}

{grinçant = squeaky}

{pincé = pinched}

{par contraste avec = in contrast to}

{haine = hate, hatred}

{haïr, abhorrer, détester = to hate, to abhor, to detest}

{De retour à = Back to}

{têtue, obstiné, opiniâtre = stubborn, obstinate}

{couler de = to flow from}

{déchirant = heartbreaking}

{facilité = ease}

{naturel = natural}

{avec tant de naturel =~}{so naturally}

{le crépuscule (}{lorsque le Soleil vient de se coucher (crépuscule du
soir) ou va se lever (crépuscule du matin)}{) = dusk, twilight}

{Se lever = to get/stand up}

{au fond, après tout = basically, at bottom}

{Plus que tout = more than anything~}

{Dans le/au fond de, dans = at the back/bottom of. Au fond de la mer}

{au fond de moi = deep inside me}

{en bas de (is~about height,~while "au fond de" is about depth)~= below,
at the bottom of}

{en fin d'après-midi = at the end of the afternoon}

{à l'infini = to infinity}

{rouvrir et refermer = reopen and close}

{se dissiper, se disperser, disparaître~}{de manière progressive} {=
dissipate, disperse, disappear}

{humeur massacrante = foul mood}

{qui s'écrit = that is written}

{se rendre compte = realize}

{au bout =~}{at the end}

\begin{itemize}
\item
  {au bout de = at the end of, in}
\end{itemize}

{s'achèver, se terminer, prendre fin = to come to an end}

{Éteindre = to put out, turn/switch~off}

{Macaron = (coconut)~macaroon (shredded coconut-based),
(Parisian)~macaron (almond flour-based)}

{Mettre:}

{Se mettre en colère, s'énerver = to get angry~}

{se mettre à = to begin to}

\begin{itemize}
\item
  {se mettre au sec = get dry}
\item
  {se mettre au service de = put oneself at the service of}
\item
  {se mettre à table = sit at table (for a meal)}
\item
  {Se mettre à sa portée = Get within reach}
\item
  {mettre à l'épreuve = to challenge, test, put to the test}
\end{itemize}

{se mettre à l'aise = to make yourself comfortable, to put at ease}

{mettre à jour = update}

{Mettre de la musique = to put some music on}

\begin{itemize}
\item
  {j'adore cette chanson mets-la jusqu'à ce que je la déteste}
\end{itemize}

{Mettre la table = to set/lay the table}

\begin{itemize}
\item
  {la table mise = the table set}
\end{itemize}

{Mettre de l'ordre = to put in order~}

mettre des affiches = to put up posters

Mettre du maquillage = to put on makeup~

Mettre tout ensemble = to put everything together~

{S'occuper = to keep oneself busy~}

{Subir = to undergo, suffer}

\begin{itemize}
\item
  {Quelles blessures as-tu subit dans le passé ?}
\end{itemize}

{Il suffit de = simply, they just have to}

{Tu serais peut-être arrivé = maybe you would have arrived, you might
have arrived (NOT you would have maybe arrived)}

{Poser =~put down}

\begin{itemize}
\item
  {Poser sur le feu = to put on the heat~}
\end{itemize}

{Technologie:}

\begin{itemize}
\item
  {télécommande = remote control}
\item
  {Lecteur = player}
\item
  {Téléphone, portable = phone, telephone}
\item
  {portable = cell phone, laptop}
\item
  {ordi(nateur) = PC}
\item
  {Souris = mouse}
\item
  {écran verrouillé = locked screen}
\item
  {clavier = keyboard}
\item
  {touche = button}
\item
  {Allumer = switch on}
\item
  {Eteindre = switch off}
\item
  {sauvegarder = to back up, save}
\item
  {le menu principal = the main menu}
\item
  {paramètres = settings}
\item
  {luminosité = brightness}
\item
  {chaîne (f.) = chain, channel}
\end{itemize}

{Une information = a piece of information~}

{À la douane = at customs~}

{Mourir de froid = to die from cold~}

{Partir en voyage = to go on a trip~}

{En faisant = while/upon/from doing}

{Se tromper dans = to make mistake in~}

{Se tromper de = to get the wrong}

{Zut = darn}

{Comment tu vas faire pour = how are you going to~}

{T'en donner = give you some~}

{Des neufs = new ones~}

{Sois-y, sois là~= be there~}

{Contrôleur/controleus~= inspector~}

{Tarif = tariff, fare}

{Finalement = in the end, after all,~}{Ultimately, eventually}

{Enfin = finally~}

{N'avoir que = to only have}

{Accessoire de mode = fashion accessories~}

{Accessoire à la mode = fashionable accessories~}

{heure:}

{Une demi-heure = half an hour~}

Une heure et demie = one and a half hours

\begin{itemize}
\item
  Il est~Une heure et demie = It is half past 1
\end{itemize}

Il est~Une heure et quart= It is quarter past 1

trois quarts d'heure = 45 minutes

La gare = (train) station~

{Je n'ai pas de = I don't have any, I haven't got any~}

{il y a:}

\begin{itemize}
\item
  {il doit y avoir = there must be}
\item
  {Il y a = there is/are}
\item
  Il {y a eu = there was/were, there has/have been}
\item
  {Il y avait eu = there had been}
\end{itemize}

{La corbeille à papier = the wastebasket, paper basket}

{La poubelle = trash/garbage can, rubbish bin, dustbin, bin}

{À/dans la poubelle = in the trash can~}

{Coffre = trunk~}

{Tournée =~}{tour,~ turn}

Partir {en tournée = to go on tour~}

C'est {ma tournée = it's my turn~}

{Ouïe = hearing~}

{Odorat = smell}

{Encore = again, yet another, still}

Yet = pourtant, cependant, toutefois~

{Au courant = aware, informed~}

{Pot = pot, jar}

{Par contre = however~}

{Les mots apparentés:}

\begin{itemize}
\item
  {La moquette = the wall-to-wall carpet}
\item
  {La carpette (petit tapis) = the rug, doormat}
\item
  {Le tapis = the rug}
\item
  {La tapisserie = the upholstery~}
\item
  {Tapissé = upholstered~}
\item
  {Bienvenue mat~= welcome mat}
\item
  {Tapis sol = floor mat}
\item
  {Couverture bébé (couverture~tapis)~= baby blanket~}
\item
  {Nappe de pique-nique (couverture~tapis)~= picnic tablecloth~}
\item
  Napperon, tapis de déjeuner = place mat, floor picnic mat
\item
  tapis de déjeuner =~Luncheon mat
\item
  Sous-main = desk mat/blotter
\item
  Le linge de plage = beach towel~
\end{itemize}

{Se salir = to get dirty,}

{Salir = to get dirty, dirty, dirty up}

{Le pot de fleurs = the flowerpot~}

{Filaire = wired}

{La lenteur = the slowness/sluggishness~}

{Le pneu~ de vélo = bike tire}

{Le pneu du vélo = bike's tire}

{S'ennuyer = to get/feel bored, is bored}

{Un puzzle de mille pièces = a thousand-piece puzzle}

{Le~Portable = the cell phone, laptop}

{Plus bref = briefer, more brief~}

{Robinet = tap, faucet, spigot}

{C'est inutile, ça ne sert à rien = it's useless~}

{Régler = to settle/sort out/solve/resolve}

{Gérer, s'occuper de~= to handle/deal with, manage}

Il a fallu faire = it had to be done, we had to do

Il me faudrait = I would need~

{Pénible = tiresome, painful~}

{Ennuyeux, agaçant, fâcheux = annoying~}

{se fâcher = to get angry}

{À cette heure-ci = at this time~}

{Au calme = quiet~}

{Klaxonner = to honk}

{Entrer dans = enter}

{En classe, en cours = during class, in the classroom~}

{Lire bien = read well (quality of one's~ability to read), read
thoroughly/carefully/properly~(thoroughness and level of care when
reading)}

{Ça va = how are you doing (American eng), how are you going (Australian
eng)}

{Faire du cerf-volant, faire voler un cerf-volant~= to fly a kite, to go
kite-flying}

{Visiter = to visit/tour}

{Le site touristique = the tourist attraction/spot/site}

{Aller en excursion, faire~ une excursion~= to go on an excursion~}

{Aller à la pêche = go fishing~}

{Faire du cerf-volant = to fly a kite}

{Tant, tellement, autant~= so much~}

{Faire des grillades, griller = to grill}

{La classe = the classroom, the group of students~}

Le~{cours = the lesson/course~}

{Accepter de = to accept~...ing}

{Il ne faut avoir aucun ... = you must not have any ...}

{Bien/beaucoup plus dur = much/far/way(slang) harder}

{Pendent:}

\begin{itemize}
\item
  {during: pendant mes vacances~}
\item
  {For: il a dormi pendant deux heures~}
\item
  {Pendant que = while, whilst}
\end{itemize}

{La pop = the pop, pop music~}

{Voler = steal, fly}

{Beaucoup de monde/gens/personnes~a/ont/ont vu = a lot of people saw}

{Raconter = to tell}

{Dire = to say}

{Quelques choses bizarres = a few bizarre things~}

{Quelque chose de bizarre = something bizarre~}

{Bizarre, étrange = bizarre, weird~}

{Répondre à = to answer~}

{Une commode = a dresser~}

{Classique = classical~}

{Tout neuf = brand new~}

{Parler de = to be about~}

{Rêver de = to dream of~}

{Faire de la danse, danser = to dance}

{Pousser = to push~}

{Instructions, consignes = instructions~}

{Crowd:}

\begin{itemize}
\item
  {Bondé(e) = crowded~}
\item
  {Dans la foule = in the crowd~}
\end{itemize}

{Participer à} {(dans)}

{Le temps file = time flies~}

{Prendre des nouvelles de qqn, demander des nouvelles à qqn=~To check in
on someone~}

{Prendre de tes nouvelles = to check in on you~}

{se passer:}

{Il se passe des choses, des choses se passent~= things happen~}

{Se passer, arriver, se produire, survenir~= to happen~}

\begin{itemize}
\item
  {ça peut arriver = It can happen}
\item
  {ça s'est bien passé = it went well}
\end{itemize}

{Bulletin (scolaire)~= report card, bulletin, (school)~newsletter (UK)}

{Tu as le droit DE~= you have the right to}

{Tu es autorisé À = you are allowed to}

{Toujours = still, always~}

{Encore = again, still}

{Rendre = turn in, hand in}

{Un prof d'espagnol = a Spanish teacher (a teacher who teaches Spanish)}

{Un prof espagnol = a Spanish teacher (a teacher who is a Spanish
citizen)}

{me donne la nausée,~}{me donne~}{envie de vomir} {= makes me nauseous}

{la nausée = nausea}

{pour le coup = for once}

{débile (Qui manque de force physique. faible, Imbécile, idiot)
=~}{stupid, feeble, dumb}

{inanité (Caractère de ce qui est vain, vide, néant) = inanity}

{Prairie,~}{pré} {= prairie, meadow, grassland}

{platitude (Carac}{tère de ce qui est uniforme, sans trait marquant,
sans attrait,} {sans originalité}{) =~}{platitude, flatness, dullness}

{fuir = to run away, to flee, to escape}

{les prévisions météorologiques =~}{weather forecast}

{toutes les nuances de = all shades of}

{niaiserie = nonesense, silliness}

{niaise = silly}

{patrie = mother/father/homeLand, native land}

{mère patrie = mother land/country}

{ça me colle à la peau (À propos de quelque chose dont on ne peut se
défaire, dont on ne peut se débarrasser, qui fait corps avec qqn, de non
détachable de la personne, qui est indissociable de qqn,} {s'unir de
manière à ne faire qu'un}{) =~}{it sticks to my skin}

{faire corps avec (}(Sens figuré) Ne faire qu'un, être en communion
avec{) = to~}{be one with}

{étouffement = choking, suffocation}

{tout au plus = at most}

{troublant = disturbing, unsettling}

{j'éprouve, je~}{ressente}{~= I feel}

{percevoir (}Comprendre, parvenir à connaître{) =~}{perceive}

{Début :}

\begin{itemize}
\item
  {En/au début de = at the~start/beginning of~}
\item
  {au tout début = at the very beginning}
\end{itemize}

{mince = thin}

{étrangeté = strangeness}

{épaisseur,~}{une couche/dimension de qqch} {= thickness, layer}

{Écouter la radio (PAS à la radio) = to listen to the radio~}

{être}~obligé de~= to be obligated to, have to~

bien = way, far, much, a great deal

Beaucoup/plein de = a lot of~

visiter de = to visit

tout seul = by my(/him ...)self, all alone (more literal)

Avec une heure d'avance = with 1 hour to spare, 1 hour ahead of
schedule/time, 1 hour early

La crème (solaire)~= sunscreen, sunscreen/sun lotion, sun cream~

À la mode = fashionable~

~Il n'y a plus que ..., il ne reste que ..., il ne reste~plus que ... =
There is (are) only ... left

Il n'y en a plus ..., il n'en reste plus ... =~There isn't (aren't) any
... left

Il n'y en a pas ... =~There isn't (aren't) any ...~

Le témoignage = testimony, witness

Expatrié = expatriate

Consignes = instructions

Exil = exile

les rapprocher (joindre, réunir) =~bring them closer

Rapprocher qqc./qqn DE qqc./qqn = Placer quelque chose ou quelqu'un plus
près de quelque chose ou de quelqu'un.

Rapprocher des choses/des personnes (l'une de l'autre, les unes des
autres) = Rendre voisin.

S'approche DE~= to get close to

Kiosquier =~~{Pers}{onne qui tie}{nt un kiosque à journaux.}

{j'ai fait des connaissances =~}{I made acquaintances}

{demander des nouvelles de = ask about}

{faire partie de = belong to, be part of}

{allumer un feu/barbecue}

{être de mauvaise humeur =~}{to be in a bad mood}

{Télé :}

\begin{itemize}
\item
  {Les émissions de télé = tv shows~}
\item
  {Présentateur (trice) = presenter, host}
\item
  {La télé réalité = reality show~}
\item
  {Feuilleton = soap opera~}
\item
  {Les séries télé = tv series~}
\item
  {Un jeu télé = game show~}
\item
  Documentaire = documentary~
\item
  Programme d'informations = news program
\item
  Un spectacle de talents~= talent show~
\item
  Zapper, passer fréquemment d'un programme de télé à un~autre au moyen
  de la télécommande~= to zap
\item
  À la télé = on tv
\end{itemize}

{passer}

{Intransitif~}

\begin{itemize}
\item
  {come by (plane, restaurant), come along (plane), pass by (plane),
  stop by (}{restaurant, not plane, someone's house/location}{)}
\item
  {Passer à la télé = to be on TV}
\item
  {Passer par = to pass by, to go down}
\end{itemize}

{Transitif~}

to spend =

dépenser (l'argent)

passer (le/du temps):

{passer} qqch {à} faire = to spend sth doing:

\begin{itemize}
\item
  passer son temps à étudier = spend his/her time studying
\end{itemize}

{y passer} qqch:

\begin{itemize}
\item
  y passer son temps
\end{itemize}

{Passer mon permis~}

{Passer un examen = to take an exam}

{Passer l'aspirateur = to vaccum (US), to~hoover (UK)}

{passer un agréable/bon séjour, bon séjour = have a nice stay}

{D}{écalage horaire = jetlag}

{on n'est pas~}{obligé de le croire = one doesn't have to believe him}

{on n'est pas~}{obligés de le croire = we don't have to believe him}

{Je pouvais = I was being able to}

{J'ai pu = I was able to}

{on se rend des petits services =~}{we do small favors}

{former entre eux une communauté = to~}{form a community among
themselves}

{(Créer)~famille élargie =~}{extended family}

{sans y être impliqué =~}{without being involved}

{Côtoyer = Au fig. Vivre à côté de (quelqu'un), être en contact avec
(quelqu'un).}

{je suis contente de le côtoyer = I'm happy to meet him}

{éberluer =}{Au fig.} {Remplir d'étonnement, causer une grande
surprise.}

{éberlué, stupéfait, Ébahi = flabbergasted, dumbfounded}{, stunned}

{le record a été battu =~}{the record has been broken}

{délire =~}{delirium}

{les veganS/les vegan}

{vegan = v}{ég}{étalien}

{pointilleux = picky, finicky}

{faire une fête,~ avoir une soir}{ée =~}{to have a party}

{en espèces (in Canada: en (argent) comptant) = in cash}

\begin{itemize}
\item
  {Je peux payer~}{en espèces (par carte)?}
\end{itemize}

{espèce = species}

{shopping:}

\begin{itemize}
\item
  {faire les magasins (for pleasure; retail therapy) = to go
  shopping,~}{shopping}{~}
\item
  {faire les courses (out of necessity) =} {to~}{go shopping,} {to~}{go
  grocery shopping,} {to~}{do the shopping}
\end{itemize}

{prendre un verre = to have/get a drink, to have/get a glass~}

{on prend un verre? = shall we go for a drink?}

{ça te/vous dit de ...? = Used for a friendly invitation to do sth}

{Je n'ai pas envie de (infinitive/noun) = I don't feel like ...ing}

{Embouteillage = traffic jam}

{Circulation = traffic~}

{zézayer (zozoter, Avoir le défaut de prononciation qui consiste à
substituer le son s {[}s{]} au son ch et le son z au son j .) = lisp}

{fin = thin}

{'adj' comme du/de la 'noun' = as 'adj' as 'noun', NOT 'adj' like
'noun'}

{fin comme du papier = as thin as paper (NOT thin like paper), thin as
paper, paper thin}

{inconcevable =~}{inconceivable}

{le mal de mer = sea sickness}

{le mal du pays = homesick}

{avoir le mal du pays = feeling homesick}

{à f}{ortiori (}{à plus forte raison) =~}{all the more}

{envier = envy}

{pays d'origine, pays natal = country of origin/birth, native country}

{capricieux =~}{capricious}

{à mesure que le temps passe =~}{as time passes,~}{as time goes by}

{poser = lay, put, put down}

{baisser le son/volume = lower the volume, put/turn the volume down}

{Baisser LA voix = lower YOUR voice~}

{toujours pas = still not}

{pas toujours = not always}

{défaire les bagages/valises = to unpack}{~}

{le sommet = the top}

{prendre le risque = to~}{take the risk}

{Prendre des risques = to take a risk}

{une tortue = a turtle}

{estimer = think, feel}

{quand même = anyway}

{vite (adv, il court vite), rapidement = fast}

{rapide (adj, il est rapide) = fast, quick}

{faire de la voile = go sailing}

{couloir = hall, hallway, corridor}

{guirlandes = tinsel,~}{fairy lights}

{aller/sortir~en boîte (de nuit)~= to go clubbing, to go to a night
club}

{discothèque,~}{boîte (de nuit)} {= disco,~nightclub}

{la nuit tombée = after dark}

{la monnaie, la devise = currency, change}

{Ça vous dérangerait de = would it bother you to, would you mind ...ing,
would it be a bother to}

{Baisser = lower, turn down~}

{Faire un barbecue = to have a barbecue}

{Bonbons = candies, sweets}

{Biscuits~= cookies, biscuits~}

{Trop (de) = too many/much of}

{Trop longtemps = too long}

{Prévoir de = to plan to}

{Il gagne bien sa vie = he makes a good living~}

{chercher = pick up}

\begin{itemize}
\item
  {Il faut aller chercher les enfants~}{à l'école}
\end{itemize}

{au coin de = at the corner of}

{Être pressé = to be in a rush}

{la voie = the track}

{le~}{casque = the helmet}

{faire du v}{é}{lo = to ride a bike, to bike}

Le père Noël, le papa Noël = the santa

traîner = hang out

se cacher la tête derrière quelque chose =~to hide one's head behind
something

faire signe à quelqu'un =~wave to someone

vue = view, vision, eyesight

fil = thread, wire, yarn (used for knitting)

faire la grasse matinée = to lie in, to have a lie in, to sleep in (to
sleep until later than usual)

à la douane =~at the customs

cahier d'activité =~activity book, workbook

sur le plateau = on the set

faire la mise à jour = update, do the update

un clou = a nail

suggérer de = to suggest

On se dépêche/presse (action)~= we hurry, we are hurrying~

On est pressés (state of being) = we are in a rush/hurry~

Les pneus de voiture = (general car) car tires

Les pneus de la voiture = (specific car) car's tires~

Demander qqch à qqn = to~ask someone for sth

Économiser, mettre de l'argent de côté, épargner~= save up~

santé = health:

Être en forme = to be in (good) shape, to be fit, to be in form
(physically)

Être en bonne santé = to be healthy (lit. to be in good health)

docteur (doctoresse)

visite médical = medical examinaiton

spécialiste (m.) médical = medical spécialist

chirurgien = surgeon

le service (hospitalier) = ward

hospitalisation

enregistrement (papiers administratifs)

les heures de visite = visiting hours

Num 15 - le samu = 999

{Faire mal à = to hurt}

\begin{itemize}
\item
  Ça fait mal : it hurts~
\item
  Faire mal à tes pieds~
\end{itemize}

{hurt oneself, get hurt~= se faire mal à, se blesser (à)}

\begin{itemize}
\item
  {to hurt your feet = te faire mal aux pieds, te blesser les
  (aux)~pieds}
\end{itemize}

{Casser/ se casser = to break}

{Bleu = bruise}

{Bandage = bandage~}

{Point = stitch~}

\begin{itemize}
\item
  {Ils m'ont fait des points au genou~}
\end{itemize}

{Saigner = to bleed}

{cicatrice = scar}

{entorse (n.) = sprain}

{se tordre (v.) = twist}{, bend}

{se fracturer =~}{fracture}

Danser, faire de la danse = to danse~

Faire du ski, skier = to ski

Couverts = hand-held instruments used for preparing, serving, and eating
food.~

\begin{itemize}
\item
  Modern sense:~Flatware/cutlery
\item
  Traditional sense: flatware doesn't contain knives, they are part of
  the cutlery.
\end{itemize}

Couverts en argent = silverware (stainless steel flatware or silver
flatware), silver cutlery~

Perdr connaissance = to faint

Plier =~ to fold

s'est mis = started

s'est mis en colère = got angry

s'inscrire à = to register for, to enroll in

Je n'en peux plus = I cannot take it anymore~

La réunion parents-profs = parent-teacher meeting~

S'informer = to inform themselves (to get informed: awkward eng), to get
their information, to~become informed~

Inscription = registration~

J'ai tant prié pour que le prof annule l'examen = I prayed so much that
the teacher MIGHT cancel the exam (to pray is often expresses by a
clause containing modals may/might depending on the tense concord. One
of the modals will/would,~ may/might, can/could corresponds more closely
to the subjunctive in French)

Tant = so much/many

Cours de théâtre = theater/drama class

Sur internet = on the internet~

Notes de cours = class/course~notes

Accepter de = to agree to~

Un responsable (n) = a person in charge (supervisor, manager, officer,
...)

Assumer la responsabilité = to take the blame

Mentionner un problème = to mention a problem~

Prendre des notes = take (some) notes~

\hfill\break


